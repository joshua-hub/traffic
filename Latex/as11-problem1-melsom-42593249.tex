\documentclass[11pt]{article} % use larger type; default would be 10pt

\usepackage[utf8]{inputenc} % set input encoding (not needed with XeLaTeX)

% PAGE DIMENSIONS
\usepackage{geometry} % to change the page dimensions
\geometry{a4paper} % or letterpaper (US) or a5paper or....
\geometry{margin=1in} % for example, change the margins to 2 inches all round
% \geometry{landscape} % set up the page for landscape
%   read geometry.pdf for detailed page layout information

\usepackage{graphicx} % support the \includegraphics command and options
\usepackage{mathtools}
\usepackage{amssymb,amsmath}
\usepackage{gensymb}
\usepackage{float}
\usepackage{epstopdf}

%PACKAGES
\usepackage{booktabs} % for much better looking tables
\usepackage{array} % for better arrays (eg matrices) in maths
\usepackage{paralist} % very flexible & customisable lists (eg. enumerate/itemize, etc.)
\usepackage{verbatim} % adds environment for commenting out blocks of text & for better verbatim
\usepackage{subfig} % make it possible to include more than one captioned figure/table in a single float

% HEADERS & FOOTERS
\usepackage{fancyhdr}
\pagestyle{fancy}
\renewcommand{\headrulewidth}{0pt}
\lhead{phys3071-as11-melsom-42593249}\chead{}\rhead{}
\lfoot{}\cfoot{}\rfoot{}

% SECTION TITLE APPEARANCE
\usepackage{sectsty}
\allsectionsfont{\sffamily\mdseries\upshape} % (See the fntguide.pdf for font help)
\setcounter{secnumdepth}{-1} 

% ToC (table of contents) APPEARANCE
\usepackage[nottoc,notlof,notlot]{tocbibind} % Put the bibliography in the ToC
\usepackage[titles,subfigure]{tocloft} % Alter the style of the Table of Contents
\renewcommand{\cftsecfont}{\rmfamily\mdseries\upshape}
\renewcommand{\cftsecpagefont}{\rmfamily\mdseries\upshape} % No bold!

% END Article customizations

\title{PHYS2041 Lab report 2\\ Single Photon Interference.}
\author{Joshua Melsom - 42593249}
\date{}

\begin{document}% GNUPLOT: LaTeX picture with Postscript


\section{Question 1}
\begin{figure}[!h]
\centering
\includegraphics[scale=0.8]{diagram.eps}
\caption{Visualisation of the FTCS scheme.}
\label{fig:appi}
\end{figure}
\section {introtuction}
Usig the notation $u^n _j = u(t_n, x_j)$ for $x_j = x_0 + j\delta x, \, j=0,1,2...,J$ and $t_n = t_0 + n\delta t, \, n=0,1,2,...N$ to approximate the PDE $\frac{\partial u}{\partial t} = -v\frac{\partial u}{\partial x}$. Starting with the approximations: \\


$\displaystyle{\frac{\partial u}{\partial t}  \bigg|_{j,n} \approx \frac{u^{n+1}_j - u^n _j}{\delta t}}$\\ [0.5em]

$\displaystyle{\frac{\partial u}{\partial x} \bigg| _{j,n} \approx \frac{u^n_{j+1} - u^n_{j-1}}{2\delta x}}$\\[0.5em]

we rearrange our PDE, $\displaystyle{\frac{\partial \rho}{\partial t} + v_{max} \frac{\partial \rho}{\partial x} \bigg ( 1-\frac{2\rho}{\rho _{max}} \bigg) = 0}$, where $\rho= \rho (x,t)$ to the form \\

$\displaystyle{\frac{\partial \rho}{\partial t} = - v_{max} \frac{\partial \rho}{\partial x} \bigg ( 1-\frac{2\rho}{\rho _{max}} \bigg)}$.\\

Now we can approximate our PDE using the listed approximations above to: \\

$\displaystyle{\frac{\rho ^{n+1}_j - \rho^n _j}{\delta t} = -v_{max} \bigg ( \frac{\rho^n_{j+1} - \rho^n_{j-1}}{2\delta x} \bigg ) \bigg (1-\frac{2 \rho^n _j}{\rho_{max}} \bigg )}$\\ [0.5em]

$\displaystyle{\rho^{n+1}_j = \rho ^n _j - v_{max}\delta t   \bigg ( \frac{\rho^n_{j+1} - \rho^n_{j-1}}{2\delta x} \bigg ) \bigg (1-\frac{2 \rho^n _j}{\rho_{max}} \bigg )}$ \\[0.5em]

And using the Lax-Friedrich approximation $u^n _j = \frac{1}{2} \bigg (  u^n _{j+1} + u^n_{j-1}  \bigg )$ and bringing $\frac{1}{2\delta x}$ outside of the bracket, we end up with:\\[0.5em]

$\displaystyle{\rho^{n+1}_j =  \frac{1}{2} \bigg (  \rho^n _{j+1} + \rho^n_{j-1}  \bigg ) - \frac{v_{max}\delta t}{2\delta x}   \big ( \rho^n_{j+1} - \rho^n_{j-1} \big ) \bigg (1-\frac{2 \rho^n _j}{\rho_{max}} \bigg )}$ \\[0.5em]




\end{document}